\documentclass{article}
\usepackage{ctex}
\usepackage{lmodern}
\title{数学建模之相关性分析}
\date{\today}
\author{ZachySun}

\usepackage{ listings}
\pagestyle{empty} 
\usepackage{amsmath}
\usepackage{geometry}
\geometry{a4paper,scale=0.8}

\begin{document}
\maketitle
\section{相关系数}

\subsection{皮尔逊相关系数}
\subsubsection{总体皮尔逊相关系数}

$$
\rho _{XY}=\frac{Cov\left( X,Y \right)}{\sigma _X\sigma _Y}
$$
$$
E\left( X \right) =\frac{\sum_{i=1}^n{X_i}}{n},E\left( Y \right) =\frac{\sum_{i=1}^n{Y_i}}{n}
$$
$$
\text{总体协方差}:Cov\left( X,Y \right) =\frac{\sum_{i=1}^n{\left( X_i-E\left( X \right) \right) \left( Y_i-E\left( Y \right) \right)}}{n}
$$
$$
\sigma _X=\sqrt{\frac{\sum_{i=1}^n{\left( X_i-E\left( X \right) \right) ^2}}{n}},\sigma _Y=\sqrt{\frac{\sum_{i=1}^n{\left( Y_i-E\left( Y \right) \right) ^2}}{n}},
$$

\subsubsection{样本皮尔逊相关系数}
$$
\text{样本}pearson\text{相关系数:}r_{XY}=\frac{Cov\left( X,Y \right)}{S_XS_Y}
$$
$$
\text{样本均值:}\bar{X}=\frac{\sum_{i=1}^n{X_i}}{n},\bar{Y}=\frac{\sum_{i=1}^n{Y_i}}{n}
$$
$$
\text{样本标准差:}S_X=\sqrt{\frac{\sum_{i=1}^n{\left( X_i-\bar{X} \right) ^2}}{n-1}},S_Y=\sqrt{\frac{\sum_{i=1}^n{\left( Y_i-\bar{Y} \right) ^2}}{n-1}}
$$

-1:线性负相关
+1:线性正相关
相关系数受异常值影响较大
相关系数衡量\textbf{线性}相关性,判断之前要画出散点图


\subsubsection{对皮尔逊相关系数进行假设检验}
(1)提出原假设以及备择假设

(2)构造统计量

(3)将检验值代入统计量

(4)绘制概率密度函数

(5)判断检验值

可以使用p值计算法

\subsection{斯皮尔曼相关系数}
X、Y为两组数据,斯皮尔曼相关系数:
$$
r_s=1-\frac{6\sum_{i=1}^n{d_{i}^{2}}}{n\left( n^2-1 \right)}
$$
$d_{i}为X_{i}与Y_{i}之间的等级差$

等级为排序后所在的位置

斯皮尔曼相关系数也被称为等级之间的皮尔逊相关系数

\subsection{Python实现相关操作}


\subsubsection{Python实现相关系数的计算}


\subsubsection{Python实现描述性统计}


\subsubsection{Python绘制各变量之间的散点图}


\subsubsection{Python实现相关系数矩阵热图绘制}


\subsection{MATLAB实现相关操作}
\subsubsection{MATLAB实现相关系数的计算}
\begin{lstlisting}[language=Matlab]

R = corrcoef(Test)

\end{lstlisting} 

\subsubsection{MATLAB实现描述性统计}
\begin{lstlisting}[language=Matlab]
    
    % 每一列的最小值
    MIN = min(Test);  
    % 每一列的最大值
    MAX = max(Test);   
    % 每一列的平均值
    MEAN = mean(Test);  
    % 每一列的中位数
    MEDIAN = median(Test);  
    % 每一列的偏度
    SKEWNESS = skewness(Test); 
    % 每一列的峰度
    KURTOSIS = kurtosis(Test); 
    % 每一列的标准差
    STD = std(Test);  
    Describe = [MIN;MAX;MEAN;MEDIAN;SKEWNESS;KURTOSIS;STD]  
    
\end{lstlisting} 

\subsubsection{MATLAB实现假设检验}
\begin{lstlisting}[language=Matlab]
    %% 假设检验
    x = -4:0.1:4;
    y = tpdf(x,28);  
    figure(1)
    plot(x,y,'-')
    grid on  
    hold on  
    % matlab可以求出临界值
    tinv(0.975,28)    %    2.0484
    plot([-2.048,-2.048],[0,tpdf(-2.048,28)],'r-')
    plot([2.048,2.048],[0,tpdf(2.048,28)],'r-')
    
    %% 计算各列之间的相关系数以及p值
    [R,P] = corrcoef(Test)
    
\end{lstlisting} 

\subsubsection{MATLAB计算斯皮尔曼相关系数}
\begin{lstlisting}[language=Matlab]
    %% 计算斯皮尔曼相关系数
    X = [3 8 4 7 2]'  
    Y = [5 10 9 10 6]'
    
    % 计算方法
    coeff = corr(X , Y , 'type' , 'Spearman')
    % 等价于:
    RX = [2 5 3 4 1]
    RY = [1 4.5 3 4.5 2]
    R = corrcoef(RX,RY)
    
    % 计算各列的斯皮尔曼相关系数
    R = corr(Test, 'type' , 'Spearman')
    
    % 对斯皮尔曼相关系数进行假设检验
    % 小样本直接进行查表
    
    % 大样本
    
    % 计算检验值
    disp(sqrt(590)*0.0301)
    % 计算p值
    disp((1-normcdf(0.7311))*2) % normcdf用来计算标准正态分布的累积概率密度函数
    
    %% 直接给出相关系数和p值
    [R,P]=corr(Test, 'type' , 'Spearman')

\end{lstlisting} 

\subsection{Excel实现相关操作}
\subsubsection{Excel实现描述性统计}
"数据"-->"数据分析"-->"描述统计"



\subsection{SPSS实现相关操作}
\subsubsection{SPSS实现描述性统计}
"分析"-->"描述统计"-->"描述"(在"选项"中可以选择统计量)



\subsubsection{SPSS绘制各变量之间的散点图}
"图形"-->"旧对话框"-->"散点图|点图"


\subsubsection{SPSS计算p值}
"分析"-->"相关"-->"双变量"

\subsection{正态分布检验}
(皮尔逊相关系数的条件之一)

(1)JB检验(样本数>30)

正态分布偏度为0,偏度为3

(2)shapiro-wilk检验(3<=样本数<=50)

使用SPSS:

"分析"-->"描述统计"-->"探索"-->"图"

\subsubsection{MATLAB实现正态分布检验}
\begin{lstlisting}[language=Matlab]
    %% 正态分布检验
    % 生成正态分布
    x = normrnd(2,3,100,1);   
    skewness(x)  %计算偏度
    kurtosis(x)  %计算峰度
    qqplot(x)
        
    % 循环检验所有列的数据
    n_c = size(Test,2);  
    H = zeros(1,6);  
    P = zeros(1,6);
    for i = 1:n_c
        [h,p] = jbtest(Test(:,i),0.05);
        H(i)=h;
        P(i)=p;
    end
    disp(H)
    disp(P)
    % h=1,不是正态分布

\end{lstlisting} 


\end{document}