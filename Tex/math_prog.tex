\documentclass{article}
\usepackage{ctex}
\usepackage{lmodern}
\title{数学建模之规划问题}
\date{\today}
\author{ZachySun}

\usepackage{listings}
\pagestyle{empty} 
\usepackage{amsmath}
\usepackage{geometry}
\geometry{a4paper,scale=0.8}

\begin{document}
\maketitle
\section{数学规划概述}
求解目标函数在一定约束下的极值

决策变量;目标函数;约束条件

\section{线性规划}

\subsection{线性规划的标准型}
$$
\min  C^TX
\\
C=\left( \begin{array}{c}
	c_1\\
	c_2\\
	\vdots\\
	c_n\\
\end{array} \right) ,X=\left( \begin{array}{c}
	x_1\\
	x_2\\
	\vdots\\
	x_n\\
\end{array} \right) ,n\text{是决策变量的个数}
\\
s.t.\begin{cases}
	Ax\leq b\\
	Aeq\cdot x=beq\\
	lb\leq x\leq ub\\
\end{cases}
$$

无下界:-inf
无上界:inf

对于MAX问题,可以通过将C改成相反数解决

\subsection{MATLAB中求解线性规划的命令}
[x,fval] = linprog[c,A,b,Aeq,beq,lb,ub,x0]

x0表示初始值,可以不给出

x为最小值处的x取值;fval表示最小值

求最大值,在fval最后加一个负号

如果不等式约束为>或<,可以进行放缩

双下标转单下标问题

\subsection{线性整数规划}
0-1规划是特殊的整数规划

[x,fval]= intlinprog[c,intcon,A,b,Aeq,Beq,lb,ub]

不能指定初始值,intcon可以指定哪些决策变量为整数

0-1规划特定上下界即可

\section{非线性规划}
非线性规划标准型

$$
\min  f\left( x \right) 
\\
s.t.\begin{cases}
	Ax\leq b, Aeq\cdot x=beq\\
	c\left( x \right) \leq 0, ceq\left( x \right) =0\\
	lb\leq x\leq ub\\
\end{cases}
$$

[x,fval]=fmincon[@fun,x0,A,b,Aeq,Beq,lb,ub,@nonlfun,option]

四种约束:

线性等式约束、非线性等式约束、线性不等式约束、非线性不等式约束

\section{最大最小化模型}

一般数学模型
$$
\underset{x}{\min}\left\{ \max \left[ f_1\left( x \right) ,f_2\left( x \right) ,...,f_m\left( x \right) \right] \right\} 
\\
s.t.\begin{cases}
	Ax\leq b\\
	Aeq\cdot x=beq\\
	C\left( x \right) \leq x\\
	Ceq\left( x \right) =0\\
	VLB\leq X\leq VUB\\
\end{cases}
$$

[x,fval] = fminimax(@fun,x0,A,b,Aeq,beq,lb,ub,@nonlfun)

\section{多目标规划问题}

转换为单目标规划问题

敏感性分析

\end{document}