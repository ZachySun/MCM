\documentclass[UTF8]{ctexart}
\title{数学建模之层次分析法}
\date{\today}
\author{ZachySun}
\usepackage{ listings}
\pagestyle{empty} 

\usepackage{geometry}
\geometry{a4paper,scale=0.8}

\begin{document}
\maketitle
\section{AHP相关理解}
\subsection{\textbf{重要概念:判断矩阵}}

层次分析法的第一步,就是要填写判断矩阵,分布对不同的指标进行判断矩阵的构造

\subsection{\textbf{重要概念:一致矩阵}}

对于一个矩阵A,一致矩阵在判断矩阵的条件下需要满足$a_{ik}=a_{ij}\times a_{jk}$

\subsection{\textbf{重要操作之一:一致性检验}}

一致性检验的主要步骤如下:

\begin{enumerate}
\item [(1)] 计算一致性指标CI
\item [(2)] 查找对应的平均随机一致性指标RI(与n阶矩阵的n有关)
\item [(3)] 计算一致性比例CR
\end{enumerate}

$CI=\frac{\lambda_{max}-n}{n-1} $

$CR=\frac{CI}{RI}$
(若$CR<0.1$,则判断矩阵的一致性可以接受)
\subsection{\textbf{重要操作之二:计算权重}}
\begin{enumerate}
\item [(1)] 算数平均法求权重:

第一步:

第二步:

第三步:

\item [(2)] 几何平均法求权重:

\item [(3)] 特征值法求权重:

\end{enumerate}

\subsection{\textbf{重要概念:权重矩阵}}

\subsection{\textbf{层次分析法步骤总结}}

\begin{enumerate}
\item [第一步] 建立层次关系图(目标层、准则层、方案层)
\item [第二步] 构造判断矩阵
\item [第三步] 一致性检验
\item [第四步] 计算权重,得到权重矩阵(可以使用excel)
\end{enumerate}

思考:层次分析法权重的得到比较主观,最好有充足的数据或者文献支持;其次,层次分析法不适用于决策层太多的情况,而且不适用于数据已知的情况


\section{Python实现有关操作}


\subsection{Python实现AHP}
\begin{lstlisting}[language=Python]
import numpy as np
import pandas as pd

class AHP:
    # 传入判断矩阵
    def __init__(self,array):
        self.array = array   
        # 获取矩阵的大小
        self.n = array.shape[0]
        # 对RI的值进行初始化
        RI = [0,0,0.52,0.89,1.12,1.26,1.36,1.41,1.46, 
                       1.49,1.52,1.54,1.56,1.58,1.59]
        self.RI = RI[self.n-1]
        
    # 特征值法计算权重向量
    def get_eig(self):
        # 获取特征值与特征向量
        eig_val ,eig_vector = 
        np.linalg.eig(self.array)
        # 得到最大的特征值
        max_val = np.max(eig_val)
        max_val = round(max_val.real, 4)
        # 得到最大特征值对应的特征向量的索引
        index = np.argmax(eig_val)
        max_vector = eig_vector[:,index]
        max_vector = max_vector.real.round(4) 
        self.max_val = max_val
        # 计算权重向量
        weight_vector = max_vector/sum(max_vector)
        weight_vector = weight_vector.round(4)
        
        print("最大的特征值: "+str(max_val))
        print("对应的特征向量为: "+str(max_vector))
        print("归一化后得到权重向量:"+str(weight_vector))
        return weight_vector
    
    def test_consitst(self):
       
        CI = (self.max_val-self.n)/(self.n-1) 
        CI = round(CI,4) 
      
        print("判断矩阵的CI值为" +str(CI))
        print("判断矩阵的RI值为" +str(self.RI))
     
        if self.n == 2:
            print("仅包含两个子因素,不存在一致性问题")
        else:
            CR = CI/self.RI 
            CR = round(CR,4)
            if  CR < 0.10 :
                print("判断矩阵的CR值为" +str(CR) + ",通过一致性检验")
                return True
            else:
                print("判断矩阵的CR值为" +str(CR) + ", 未通过一致性检验")
                return False



\end{lstlisting} 


\section{Matlab实现有关操作}

\begin{lstlisting}[language=Matlab]
%% AHP
%% 层次分析法关键步骤
% 构造判断矩阵
% 一致性检验
%  计算权重矩阵
clear;
clc;
    
%% 输入判断矩阵
X = [] ;
    
%% 算数平均法
% 1.矩阵的每一个元素均除以其所在列的和
X_sum1 = sun(X)
n = size(X, 1)
X_sum2 = repmat(X_sum1, n, 1) 
X_c =  X./X_sum2
% 2.按照行进行求和
sum(X_c, 2)
% 3.将此时得到的向量每个元素除以n
disp('算数平均法求得的权重为:');
disp(sum(X_c,2) / n)
    
%% 几何平均法
clc;
% 1.按行相乘
X_prod = prod(X,2)
    
% 2.对新向量的元素开n次方
X_prod_n = X_prod .^ (1/n)
    
% 3.将此时得到的向量除以向量的和
disp('几何平均法求得的权重为:');
disp(X_prod_n ./ sum(X_prod_n))
    
%% 特征值法
clc;
% 1.求矩阵最大的特征值以及对应的特征向量
[V,D] = eig(X)  
max_eig = max(max(D))
[r,c] = find(D == max_eig , 1) 
    
% 2.对求出的特征向量进行归一化
disp('特征值法求得的权重为:');
disp( V(:,c) ./ sum(V(:,c)) )
    
%% 判断是否通过一致性检验
clc;
CI = (max_eig - n) / (n-1);        % 求解一致性指标
% RI(2)有可能需要更改
RI=[0 0 0.52 0.89 1.12 1.26 1.36 1.41 1.46 
1.49 1.52 1.54 1.56 1.58 1.59];   
% 求解一致性比例
CR=CI/RI(n);                              

disp('CI=');disp(CI);
disp('CR=');disp(CR);
if CR<0.10
    disp('CR<0.10,该判断矩阵通过一致性检验');
else
    disp('CR>=0.10,该判断矩阵未通过一致性检验');
end

\end{lstlisting} 

\end{document}
