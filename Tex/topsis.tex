\documentclass{article}
\usepackage{ctex}
\usepackage{lmodern}
\title{数学建模之topsis法}
\date{\today}
\author{ZachySun}

\usepackage{ listings}
\pagestyle{empty} 
\usepackage{amsmath}
\usepackage{geometry}
\geometry{a4paper,scale=0.8}

\begin{document}
\maketitle
\section{topsis相关理解}

构造计算评分的公式:

$ \frac{x-min}{max-min} $
变形:

$ \frac{x-min}{(max-x)+(x-min)} $

评价指标分类:

(1)极大型指标

(2)极小型指标

(3)中间型指标

(4)区间型指标

**指标正向化:

(1)极小型指标->极大型指标: $max-x$

(2)中间型指标->极大型指标:

$M=max{|x_{i}-x_{best}|}, \tilde{x_{i}}=1-\frac{|x_{i}-x_{best}|}{M}$

(3)区间型指标->极大型指标:
最佳区间为[a,b]

$$
M=\max \left\{ a-\min \left\{ x_i \right\} ,\max \left\{ x_i \right\} -b \right\} , \tilde{x}_i=\begin{cases}
	1-\frac{a-x}{M}, x<a\\
	1,    a\leq x<b\\
	1-\frac{x-b}{M}, x>b\\
\end{cases}
$$

**标准化处理:

$$
X
=\begin{bmatrix}
x_{11}  &  x_{12}  & \cdots\ &x_{1m}\\
x_{21}  &  x_{22}  & \cdots\ & x_{2m}\\
\vdots   & \vdots & \ddots  & \vdots  \\
x_{n1} & x_{n2}  & \cdots\ & x_{nm}\\
\end{bmatrix}
$$

将对其标准化的矩阵记为T,则$t_{ij}=\frac{x_{ij}}{\sqrt{\sum_{i=1}^{n} x_{ij}^{2} } } $

**拓展:加上权重

\section{Matlab实现有关操作}
\textbf{主函数如下:}
\begin{lstlisting}[language=Matlab]
    %% topsis
    %% 加载输入矩阵X
    clear;clc
    load % X.mat
    
    %% 正向化
    % 获取输入矩阵的行与列
    [n.m] = size(X);
    % 需要正向化处理的指标所在的列
    located = [];
    % 这些列对应的处理方式(1:极小型 ;2:中间型;3:区间型)
    func = [];
    % 将以上向量输入正向化函数Positivization
    % 传入参数为:需要正向化处理的列向量;
    %对应的处理方式;正向化处理的哪一列
    for i = 1 : size(located,2)
        X(:,located(i)) = 
        Positivization(X(:,located(i)),func(i),located(i));
    end
    
    disp('正向化后的矩阵如下:')
    disp(X)
    
    %% 标准化
    Z = X ./ repmat(sum(X.*X) .^ 0.5, n, 1);
    disp('最终得到的标准化矩阵Z= ')
    disp(Z)
    
    %% 添加权重
    % 1:添加 ;2:不添加
    Judge = input('是否需要增加权重: '); 
    if Judge == 1
        % 输入权重向量
        weigh = input(['请输入权重向量:']);
        flag = 0;  
        while flag == 0 
            if abs(sum(weigh) - 1)<0.000001 && 
            size(weigh,1) == 1 && size(weigh,2) == m   
                 flag =1;
            else
                weigh = input('error ');
            end
        end
    else
        weigh = ones(1,m) ./ m ; 
    end
    
    
    %% 计算与最大、最小值之间的距离以及评分
    % 与最大值的距离组成的向量
    D_max = sum([(Z - repmat(max(Z),n,1)) .^ 2 ] 
    .* repmat(weigh,n,1) ,2) .^ 0.5;
    % 与最小值的距离组成的向量
    D_min = sum([(Z - repmat(min(Z),n,1)) .^ 2 ] 
    .* repmat(weigh,n,1) ,2) .^ 0.5;
    % 得到未归一化的评分
    S = D_min ./ (D_max+D_min); 
    disp('最终得到的评分如下:')
    stand_S = S / sum(S);
    % 对评分进行排序
    [S_sort,index] = sort(stand_S ,'descend');


\end{lstlisting} 
\textbf{正向化函数如下:}
\begin{lstlisting}[language=Matlab]
    function [x_positive] = Positivization(x,func,i)
    if type == 1  
        x_positive = Min2Max(x);  
    elseif type == 2  
        best = input('请输入最佳值 :');
        x_positive = Mid2Max(x,best);
    elseif type == 3  
        a = input('区间下界: ');
        b = input('区间上界: '); 
        x_positive = Inter2Max(x,a,b);
    else
        disp('指标出错')
    end
end

function [x_positive] = Min2Max(x)
    x_positive = max(x) - x; 
end

function [x_positive] = Mid2Max(x, best)
    M = max(abs(x - best)); 
    x_positive = 1 - abs(x - best) / M; 
end

function [x_positive] = Inter2Max(x,a,b)
    r = size(x,1); 
    M = max([a - min(x)], max(x) - b); 
    x_positive = zeros(r , 1);  
    
    for i = 1 : r 
        if x(i) < a 
            x_positive(i) = 1 - (a - x(i)) / M;
        elseif x(i) > b 
            x_positive(i) = 1 - (x(i) - b) / M;
        else 
            x_positive(i) = 1;
        end
    end
end


\end{lstlisting} 

\end{document}