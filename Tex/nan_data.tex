\documentclass{article}
\usepackage{ctex}
\usepackage{lmodern}
\title{数学建模之缺失值处理}
\date{\today}
\author{ZachySun}

\usepackage{ listings}
\pagestyle{empty} 
\usepackage{amsmath}
\usepackage{geometry}
\geometry{a4paper,scale=0.8}

\begin{document}
\maketitle

\section{插值算法}
\textbf{插值方法}

1.多项式插值

高次插值会产生龙格现象

2.分段插值***

3.插值法原理如下:

设有n+1个互不相同的节点$(x_{i},y_{i})$,(i=0,1,2,...,n),
则存在唯一的多项式:
$$
L_n\left( x \right) =a_0+a_1x+a_2x^2+...+a_nx^n
$$
使得$L_n\left( x_j \right) =\,\,y_j$,(j=0,1,2,...,n)

4.拉格朗日插值法**

5.牛顿插值法**

6.埃尔米特插值**

最常用:分段三次埃尔米特插值

7.三次样条插值**

注:插值算法还可以用于预测
\section{插值算法的MATLAB实现}

\begin{lstlisting}[language=Matlab]
%% nan_data
%% 分段三次埃尔米特插值
x = -pi:pi; 
y = sin(x); 
new_x = -pi:0.15:pi;
p = pchip(x,y,new_x);
figure(1); 
plot(x, y, 'o', new_x, p, 'r-')
    
%% 三次样条插值
x = -pi:pi; 
y = sin(x); 
new_x = -pi:0.15:pi;
% 分段三次埃尔米特插值
p1 = pchip(x,y,new_x);   
% 三次样条插值
p2 = spline(x,y,new_x);  
figure(2);
plot(x,y,'o',new_x,p1,'r-',new_x,p2,'g-')
legend('样本点','三次埃尔米特插值','三次样条插值','Location','SouthEast')   

\end{lstlisting} 

\section{Python缺失值处理}
\begin{lstlisting}[language=Python]
    #!/usr/bin/env python
    # coding: utf-8
    
    import pandas as pd
    import numpy as np
    from numpy import nan
    
    # 生成含缺失值的数据
    data=pd.DataFrame(np.arange(3,35,1).reshape(8,4),columns=list('abcd'))
    data.iloc[1:2,:3]=nan
    data
    #     1.检查缺失值
    
    
    data.isnull()
    
    data.notnull()
    
    data.isnull().sum()
    
    data.info()
    #     2.缺失值填充
    
    data_1 = data.fillna(0)
    data_1
    
    data_2 = data.fillna("$")
    data_2
    
    data['a']=data['a'].fillna(data['a'].sum())
    data
    # In[19]:
    # 插值算法
    df = pd.DataFrame([(0.0, np.nan, -1.0, 1.0),
                       (np.nan, 3.0, np.nan, np.nan),
                       (2.0, 3.0, np.nan, 9.0),
                       (np.nan, 4.0, -4.0, 15.0)],
                      columns=list('abcd'))
    df
    
    df.interpolate(method='linear', limit_direction='forward', axis=0)
    
    df.interpolate(method='linear', axis=0)
    
    df['d'].interpolate(method='spline', axis=0, order=2)
    



\end{lstlisting} 


\end{document}