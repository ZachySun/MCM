\documentclass{article}
\usepackage{ctex}
\usepackage{lmodern}
\title{数学建模之回归分析}
\date{\today}
\author{ZachySun}

\usepackage{ listings}
\pagestyle{empty} 
\usepackage{amsmath}
\usepackage{geometry}
\geometry{a4paper,scale=0.8}

\begin{document}
\maketitle

\section{回归分析分类}
\textbf{回归分析:}

研究X与Y的相关性

\textbf{常见的回归分析的类别:}

(1)线性回归

(2)0-1回归

(3)定序回归

(4)计数回归

(5)生存回归

\section{数据分类}
1.横截面数据:

某一时间点收集的不同对象的数据

2.时间序列数据:

同一对象不同时间的数据

3.面板数据:

综合横截面数据以及时间序列数据

\section{一元线性回归}

\subsection{基本概念}

内生性会导致回归系数估计不准确,不满足无偏性以及一致性

弱化内生性:将解释变量分为核心解释变量和控制变量

只需满足核心解释变量与扰动项不相关即可

\subsection{四类模型及回归系数的解释}
(1)一元线性分析
$$
y=a+bx+u
$$

x每增加1个单位,y平均变化b个单位

(2)双对数模型
$$
\ln y=a+b\ln x+u
$$

x每增加1%,y平均变化b%

(3)半对数模型
$$
y=a+b\ln x
$$

x每增加1%,y平均变化b/100个单位

(4)半对数模型
$$
\ln y=a+bx
$$

x每增加1个单位,y平均变化(100b)%

\subsection{虚拟变量}



\subsection{含有交互项的自变量}

两个自变量之间可能会存在交互效应,可以通过求偏导进行分析

\subsection{Excel,stata操作}
\textbf{Stata}

导入:文件--导入--excel保存

后缀文件: .do

定量数据:summarize(sum)用于进行描述性统计

定性数据:tabulate(tab)  gun(可选参数,生成对应的虚拟变量)

\textbf{回归:regress(reg)}

SSR SSE SST

df:自由度

Adj R squared: 调整后的R方

联合显著性检验 F()   Prob>F  该两项重点关注

Coef:回归系数

Std.Err: 标准误

P>|t|:显著性检验,小于0.01(0.05),表示该回归系数显著的异于0

[95% Conf. Interval]:区间估计,置信区间

对于定性数据,需要使用虚拟变量进行回归

与其中一个虚拟变量进行比较分析即可

\textbf{Excel}

数据透视表

********************************************

回归分为解释性回归以及预测性回归

预测性回归更关注拟合优度

为了更精确的研究评价量的重要因素,可以考虑使用标准化回归系数

regress      ,beta

重点关注标准化回归系数

\section{异方差}

球形扰动项:

满足”同方差“和"无自相关"两个条件

横截面数据容易出现异方差问题

时间序列数据容易出现自相关问题

存在异方差的结果:

(1)OLS估计出来的回归系数仍然无偏、一致

(2)假设检验失效

(3)OLS不再是最优线性无偏估计量

解决方法:

(1)OLS+稳健的标准误

regress ,robust

(2)GLS

检验异方差的方法

stata:

rvfplot(残差与拟合值)

rvpplot x()

概率密度估计图

kdensity(核估计)

异方差的假设检验:

(1)BP检验 
estat hettest ,rhs iid

p值小于0.05,说明95%的置信水平下拒绝原假设,即我们认为扰动项存在异方差

(2)怀特检验
estat imtest,while

\section{多重共线性}

检验多重共线性:

方差膨胀因子VIF

若VIF>10,则有严重的多重共线性

estat VIF

\section{逐步回归分析}

向前逐步回归:增加自变量,显著则加入回归模型

stepwise regress ,pe(\#1)(p<\#1,加入模型)

向后逐步回归:剔除自变量,剔除最没有解释性的自变量

stepwise regress ,pr(\#2)(p>=\#2,剔除模型)

\end{document}